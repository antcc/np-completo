\documentclass[11pt,a4paper]{article}

% Packages
\usepackage[utf8]{inputenc}
\usepackage[spanish, es-tabla]{babel}
\usepackage{caption}
\usepackage{listings}
\usepackage{adjustbox}
\usepackage{enumitem}
\usepackage{boldline}
\usepackage{amssymb, amsmath}
\usepackage[margin=1in]{geometry}
\usepackage{changepage}

% Logic gates
\usepackage{tikz}
\usetikzlibrary{arrows,shapes.gates.logic.US,shapes.gates.logic.IEC,calc}

% Meta
\title{Modelos Avanzados de Computación \\ \Large{NP-completitud}}
\author{Daniel Pozo Escalona \\ Antonio Coín Castro}
\date{\today}

% Custom
\setlength\parindent{0pt}
\newenvironment{sol}{\begin{adjustwidth}{1em}{}}{\end{adjustwidth}}

\begin{document}
\maketitle

\textbf{Problema 12 e)}. Un circuito lógico es un grafo con entradas binarias, puertas AND, OR y NOT y una salida binaria. El problema \verb|CSAT| es dado un circuito lógico, determinar si existe una entrada
que hace que la salida sea verdadera. Probar que \verb|CSAT| es NP-completo.\\

\begin{sol} \textit{Solución.} En primer lugar probamos que \verb|CSAT| está en NP. Para ello, consideramos el algoritmo que consiste en generar de forma no determinista una entrada (una sucesión de valores de verdad), y comprobar si la salida que proporciona el circuito para esa entrada es verdadera. Claramente este algoritmo no determinista es polinómico en función del tamaño de la entrada y resuelve el problema tanto en los casos positivos como en los negativos. Para ver que de hecho \verb|CSAT| es NP-Completo, reducimos \verb|3-SAT| a él.

\subsection*{3-SAT $\mathbf{\propto}$ CSAT}

Dada una instancia de \verb|3-SAT| formada por un conjunto $U=\{x_0, \dots, x_{n-1}\}$ de variables junto con un conjunto $C$ de $m$ cláusulas, cada una consistente en una disyunción de exactamente tres literales en las variables de $U$, construimos por cada cláusula una puerta OR cuyas entradas son las variables involucradas, eventualmente negadas mediante una puerta NOT en caso de que aparecieran con el símbolo $\lnot$ delante en la fórmula. Las salidas de todas las puertas OR van a parar a una puerta AND final que las combina todas, y cuya salida es la salida del circuito. Notemos que si queremos obligar a que cada puerta lógica tenga únicamente dos entradas podemos emplear dos puertas OR para cada cláusula y $m-1$ puertas AND al final, encadenándolas correctamente.\\

Es decir, si denotamos indistintamente $x^*_i$ a $x_i$ ó $\lnot x_i$, y $C = \{ c_1, \dots, c_m\}$ con $c_i = x^*_{i1} \lor x^*_{i2}\lor x^*_{i3}$, la satisfabilidad de $C$ equivale a la de

\[
\bigwedge_{i=1}^m (x^*_{i1} \lor x^*_{i2} \lor x^*_{i3}),
\]

y de esta forma el circuito lógico resultante de la reducción se puede escribir (en notación prefija) como:

\[
\text{AND}(\text{OR}(X_{11},X_{12},X_{13}), \dots, \text{OR}(X_{m1}, X_{m2}, X_{m3})),
\]

donde $X_{ih} = x_i$ ó $X_{ih} = \text{NOT}(x_i)$ dependiendo de si era $x^*_{ih} = x_{ih}$ ó $x_{ih}^* = \lnot x_{ih}$.\\

La equivalencia entre ambos problemas es clara: si hay una asignación de valores de verdad que hace verdaderas todas las cláusulas de la instancia de \verb|3-SAT|, esa misma asignación hará que la salida del circuito construido sea verdadera, y viceversa.\\

Además, esta reducción se realiza en espacio logarítmico, pues consiste únicamente en ir leyendo la entrada e ir escribiendo en la salida, sustituyendo los símbolos tal y como se ha explicado. Si suponemos, por ejemplo, que la entrada al problema \verb|3-SAT| viene dada en la forma:

\[
  n_{0} c n_{1} c n_{2} c \dots c n_{3m-3} c n_{3m-2} c n_{3m-1}
\]

con $n_{i}$ un número entre 0 y $n-1$ en binario, más un bit de negación delante, y entendiendo que el conjunto de cláusulas es $C = \{ x^*_{n_{3i}}\lor x^*_{n_{3i+1}} \lor x^*_{n_{3i+2}} : i=0,\dots,m-1 \}$, y que la entrada a \verb|CSAT| que calcula la reducción es:

\[
  \underbrace{n_{0}c\dots cn_{3m-1}}_\text{nodos}\ X\ \underbrace{n_{0}e\texttt{or}_{1}cn_{1}e\texttt{or}_{1}cn_{2}e\texttt{or}_{1}c\ \dots\ cn_{3m-1}e\texttt{or}_{m}}_\text{disyunciones entre los nodos}\ X \ \underbrace{\texttt{or}_{1}e\texttt{and}c\dots \texttt{or}_{m}e\texttt{and}}_\text{conjunciones}
\]

Notamos que los símbolos \textit{c,e} y \textit{X} son separadores. Para calcular esta entrada a partir de la primera solo son necesarios un número constante de contadores para almacenar, a lo más, $m$, y que por tanto solo requieren de espacio logarítmico en la entrada.

\subsection*{Ejemplo}

Sea $U = \{x_0,x_1,x_2,x_3\}$ y $C = \{ x_0 \lor \lnot x_1 \lor x_2, \lnot x_0 \lor x_1 \lor x_3 \}$. La entrada codificada para la reducción sería

\[
000c101c010c100c001c011
\]

y por tanto la salida calculada que representa el circuito lógico asociado es
\[
000c101c010c100c001c011 \ X 
\]
\[
000e\texttt{or}_{1}c101e\texttt{or}_{1}c010e\texttt{or}_{1}c100e\texttt{or}_{2}c001e\texttt{or}_{2}c011e\texttt{or}_{2}\ X
\]
\[
 \texttt{or}_{1}e\texttt{and}c\texttt{or}_2\texttt{and}
\]

El circuito lógico que pintaríamos sería el siguiente:

%% Circuito de prueba
\vspace{2em}
\thispagestyle{empty}
\tikzstyle{branch}=[fill,shape=circle,minimum size=3pt,inner sep=0pt]
\begin{center}
\begin{tikzpicture}[label distance=2mm]

    \node (x3) at (0,0) {$x_3$};
    \node (x2) at (1,0) {$x_2$};
    \node (x1) at (2,0) {$x_1$};
    \node (x0) at (3,0) {$x_0$};

    \node[not gate US, draw, rotate=-90] at ($(x1)+(0.5,-1)$) (Not1) {};
    \node[not gate US, draw, rotate=-90] at ($(x0)+(0.5,-1)$) (Not0) {};
    \node[or gate US, draw, logic gate inputs=nnn] at ($(x0)+(2,-2)$) (Or1) {};
    \node[or gate US, draw, logic gate inputs=nnn] at ($(Or1)+(0,-1)$) (Or3) {};
    \node[and gate US, draw, logic gate inputs=nn, anchor=input 1] at ($(Or3.output)+(1,0.5)$) (And1) {};

    \foreach \i in {1,0}
    {
        \path (x\i) -- coordinate (punt\i) (x\i |- Not\i.input);
        \draw (punt\i) node[branch] {} -| (Not\i.input);
    }
    
    %\draw (x3 |- Or1.input 1) node[branch] {} -- (Or1.input 1);
  
    
    \draw (x0) |- (Or1.input 1);
    \draw (Not1.output) |- (Or1.input 2);
    \draw (x2) |- (Or1.input 3);
    
     \draw (Not0.output) |- (Or3.input 1);
     \draw (x1) |- (Or3.input 2);
     \draw (x3) |- (Or3.input 3);

    %\draw (Not0.output |- Or1.input 3) node[branch] {} -- (Or1.input 3);
    \draw (Or1.output) -- (And1.input 1);
    \draw (Or3.output) -- (And1.input 2);
    \draw (And1.output) -- ([xshift=0.5cm]And1.output) node[above] {$f$};
\end{tikzpicture}
\end{center}

\end{sol}

\end{document}