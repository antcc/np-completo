\documentclass[11pt,a4paper]{article}

% Packages
\usepackage[utf8]{inputenc}
\usepackage[spanish, es-tabla]{babel}
\usepackage{caption}
\usepackage{listings}
\usepackage{adjustbox}
\usepackage{enumitem}
\usepackage{boldline}
\usepackage{amssymb, amsmath}
\usepackage[margin=1in]{geometry}
\usepackage{changepage}

% Logic gates
\usepackage{tikz}
\usetikzlibrary{arrows,shapes.gates.logic.US,shapes.gates.logic.IEC,calc}

% Meta
\title{Modelos Avanzados de Computación \\ \Large{NP-completitud}}
\author{Daniel Pozo Escalona \\ Antonio Coín Castro}
\date{\today}

% Custom
\setlength\parindent{0pt}
\newenvironment{sol}{\begin{adjustwidth}{1em}{}}{\end{adjustwidth}}

\begin{document}
\maketitle

\textbf{Problema 12 e)}. Un circuito lógico es un grafo con entradas binarias, puertas AND, OR y NOT y una salida binaria. El problema \verb|CSAT| es dado un circuito lógico, determinar si existe una entrada
que hace que la salida sea verdadera. Probar que \verb|CSAT| es NP-completo.\\

\begin{sol} \textit{Solución.} En primer lugar probamos que \verb|CSAT| está en NP. Para ello, consideramos el algoritmo que consiste en generar de forma no determinista una entrada (una sucesión de valores de verdad), y comprobar si la salida que proporciona el circuito para esa entrada es verdadera. Claramente este algoritmo no determinista es polinómico en función del tamaño de la entrada y resuelve el problema tanto en los casos positivos como en los negativos. Para ver que de hecho \verb|CSAT| es NP-Completo, reducimos \verb|3-SAT| a él.

\subsection*{3-SAT $\mathbf{\propto}$ CSAT}

Dada una instancia de \verb|3-SAT| formada por un conjunto $U=\{x_1, \dots, x_n\}$ de variables junto con un conjunto $C$ de $m$ cláusulas en forma normal conjuntiva (CNF) y cada una con tres variables, construimos por cada cláusula una puerta OR cuyas entradas son las variables involucradas, eventualmente negadas mediante una puerta NOT en caso de que aparecieran con el símbolo $\lnot$ delante en la fórmula. Las salidas de todas las puertas OR van a parar a una puerta AND final que las combina todas, y cuya salida es la salida del circuito. Notemos que si queremos obligar a que cada puerta lógica tenga únicamente dos entradas podemos emplear dos puertas OR para cada cláusula y $m-1$ puertas AND al final, encadenándolas correctamente.\\

Es decir, si denotamos indistintamente $x^*_i$ a $x_i$ ó $\lnot x_i$, e indicamos con un segundo índice la pertenencia de una variable a una cierta cláusula, podemos escribir

\[
C = \left\{ \bigwedge_{h=1}^m (x^*_{ih} \lor x^*_{jh} \lor x^*_{kh}) : i, j, k \in \{1,\dots,n \} \right\},
\] 

y de esta forma el circuito lógico resultante de la reducción se puede escribir (en notación prefija) como:

\[
\text{AND}(\text{OR}(X_{i1},X_{j1},X_{k1}), \dots, \text{OR}(X_{im}, X_{jm}, X_{km})),
\]

donde $X_{ih} = x_i$ ó $X_{ih} = \text{NOT}(x_i)$ dependiendo de si era $x^*_{ih} = x_{ih}$ ó $x_{ih}^* = \lnot x_{ih}$.\\

La equivalencia entre ambos problemas es clara: si hay una asignación de valores de verdad que hace verdaderas todas las cláusulas de la instancia de \verb|3-SAT|, esa misma asignación hará que la salida del circuito construido sea verdadera, y viceversa.\\

Además, esta reducción se realiza en espacio logarítmico, pues consiste únicamente en ir leyendo la entrada e ir escribiendo en la salida, sustituyendo los símbolos tal y como se ha explicado sin usar memoria adicional.

\subsection*{Ejemplo}

%% Falta poner el ejemplo


%% Circuito de prueba
\vspace{2em}
\thispagestyle{empty}
\tikzstyle{branch}=[fill,shape=circle,minimum size=3pt,inner sep=0pt]
\begin{center}
\begin{tikzpicture}[label distance=2mm]

    \node (x3) at (0,0) {$x_3$};
    \node (x2) at (1,0) {$x_2$};
    \node (x1) at (2,0) {$x_1$};
    \node (x0) at (3,0) {$x_0$};

    \node[not gate US, draw, rotate=-90] at ($(x2)+(0.5,-1)$) (Not2) {};
    \node[not gate US, draw, rotate=-90] at ($(x1)+(0.5,-1)$) (Not1) {};
    \node[not gate US, draw, rotate=-90] at ($(x0)+(0.5,-1)$) (Not0) {};

    \node[or gate US, draw, logic gate inputs=nnn] at ($(x0)+(2,-2)$) (Or1) {};
    \node[or gate US, draw, logic gate inputs=nnnn] at ($(Or1)+(0,-1)$) (Or2) {};
    \node[or gate US, draw, logic gate inputs=nnn] at ($(Or2)+(0,-1)$) (Or3) {};
    \node[xor gate US, draw, logic gate inputs=nn] at ($(Or3)+(0,-1)$) (Xor1) {};
    \node[and gate US, draw, logic gate inputs=nn, anchor=input 1] at ($(Or3.output)+(1,0)$) (And1) {};
    \node[nor gate US, draw, logic gate inputs=nn, anchor=input 1] at ($(Or2.output -| And1.output)+(1,0)$) (Nor1) {};
    \node[and gate US, draw, logic gate inputs=nn, anchor=input 1] at ($(Or1.output -| Nor1.output)+(1,0)$) (And2) {};

    \foreach \i in {2,1,0}
    {
        \path (x\i) -- coordinate (punt\i) (x\i |- Not\i.input);
        \draw (punt\i) node[branch] {} -| (Not\i.input);
    }
    \draw (x3) |- (Or2.input 1);
    \draw (x3 |- Or1.input 1) node[branch] {} -- (Or1.input 1);
    \draw (x2) |- (Xor1.input 1);
    \draw (x2 |- Or3.input 1) node[branch] {} -- (Or3.input 1);
    \draw (Not2.output) |- (Or2.input 2);
    \draw (x1) |- (Or3.input 2);
    \draw (x1 |- Or1.input 2) node[branch] {} -- (Or1.input 2);
    \draw (Not1.output) |- (Xor1.input 2);
    \draw (Not1.output |- Or2.input 3) node[branch] {} -- (Or2.input 3);
    \draw (x0) |- (Or2.input 4);
    \draw (Not0.output) |- (Or3.input 3);
    \draw (Not0.output |- Or1.input 3) node[branch] {} -- (Or1.input 3);
    \draw (Or3.output) -- (And1.input 1);
    \draw (Xor1.output) -- ([xshift=0.5cm]Xor1.output) |- (And1.input 2);
    \draw (Or2.output) -- (Nor1.input 1);
    \draw (And1.output) -- ([xshift=0.5cm]And1.output) |- (Nor1.input 2);
    \draw (Or1.output) -- (And2.input 1);
    \draw (Nor1.output) -- ([xshift=0.5cm]Nor1.output) |- (And2.input 2);
    \draw (And2.output) -- ([xshift=0.5cm]And2.output) node[above] {$f_1$};
\end{tikzpicture}
\end{center}

\end{sol}

\end{document}